  \subsection{Hipótesis planteadas}

    \subsubsection*{Experimento 1: Isoterma según granularidad - Contante}
    Para realizar este experimento, se calcula la isoterma en un sistema donde la temperatura de la pared externa es constante, es decir, no varía, y la granularidad es muy alta. Suponemos que esta es la isoterma real (la más cercana a la realidad). 

    Se planea observar como varía la isoterma cuando se modifica la granularidad de los radios (sin cambiar la de los ángulos), o la de los ángulos(sin cambiar la de los radios).

    Se espera que cuanto menor sea la granularidad de la discretización, más alejada estará la isoterma calculada de la real. Si tomamos menos particiones sobre radios, la distancia entre ellos aumenta. Además, como el cálculo de la isoterma supone que la temperatura entre dos de ellos, uno más lejano y otro más cercano a la pared, decrece en forma lineal, el error es mayor. Si tomamos menos particiones sobre ángulos, existen puntos que no van a ser considerados en el cálculo de la isoterma, por lo tanto, el error también es mayor.

    \subsubsection*{Experimento 2: Isoterma según granularidad - seno}
    Este caso es igual al anterior pero las temperaturas de la pared externa del alto horno varian, no son iguales. La hipótesis es la misma que en el caso anterior.


  	\subsubsection*{Experimento 3: Índice de peligrosidad según granularidad}
  	El objetivo de este experimento es observar el cambio del índice de peligrosidad en una instancia particular del horno al ir modificando su discretización, pero manteniendo su temperatura. 

  	Se pretende generar una instancia con un sector de la pared externa donde la temperatura es notablemente mayor para poder notar facilmente el cambio en la posición de la isoterma y su respectivo valor de peligrosidad. Luego, se crean instancias similares con menor granularidad de ángulos y sin modificar las temperaturas. 

    El resultado esperado es que el índice de peligrosidad disminuya o se mantenga a medida que la granularidad de la discretización es menor. Esto sucede ya que al tomar menos puntos, es posible que el valor considerado al calcular el índice en una instancia ya no sea tomado en cuenta al calcularlo en otra instancia con menos particiones.

    \subsubsection*{Experimento 4: Índice de peligrosidad según ancho de la pared}
    Se busca estudiar la variación de la medida de peligrosidad con respecto al grosor de la pared del alto horno, sin modificar las temperaturas de la pared externa.

    La hipótesis de este experimento es que a medida que el grosor de la pared disminuye, el índice de peligrosidad aumenta. Esto se debe a que al dejar las temperaturas constantes, la isoterma se encuentra cada vez más cerca de la pared externa.

  	\subsubsection*{Experimento 5: Tiempo según número de instancias}
  	En este experimento se generan distintas instancias para un mismo sistema, con el objetivo de comparar el tiempo de ejecución al resolverlas mediante el método de eliminación gaussiana y la factorización LU. 

  	Para ello, se crean sistemas con diferentes cantidades de instancias y se comparan los tiempos de ejecución de los mismos para cada uno de los métodos. Con el fin de acercarse a los valores reales y descartar posibles falsos resultados, se ejecuta la resolución de un mismo sistema una determinada cantidad de veces y se calcula el promedio de los tiempos medidos.

    Conjeturamos que la diferencia entre los tiempos de ejecución de los dos métodos aumenta con la cantidad de instancias. 

    Al considerar un mismo sistema con diferentes instancias el método de eliminación gaussiana resulta más lento ya que resuelve cada una de ellas individualmente. En cambio, el método de factorización LU reutiliza la matriz L ya calculada para resolver cada una de las instancias.

    Destacamos que al considerar pocas instancias, factorización LU es más lenta que eliminación gaussiana, por una constante. Esto sucede ya que no se puede sacar ventaja de guardar la matriz L.


  	\subsubsection*{Experimento 6: Tiempo según granularidad}
  	El objetivo de este experimento es observar como varían los tiempos de ejecución al resolver sistemas con mismas temperaturas pero diferente granularidad, y modificando el método utilizado.
  	
    Hacemos el mismo experimento dos veces, uno para cada uno de los métodos nuemricos. Manteniendo las temperaturas constantes, variamos cantidad de particiones de los ángulos dejando la de los radios intacta. Calculamos la solución del sistema para cada una de las distintas granularidades y comparamos los tiempos que toman las diferentes ejecuciones. Repetimos esto mismo, ahora variando las particiones de los radios. Luego tomamos el otro método numérico y volvemos a iniciar el experimento. 

  	La complejidad del algoritmo es cúbico en el tamaño de la entrada. Como solo variamos la granularidad del radio o del ángulo (de jando el otro constante) entonces en ambos casos suponemos que el tiempo de ejecucion va a ser cubico con relacion a la granularidad de la entrada.


    \subsubsection*{Experimento 7: Isoterma según temperatura en pared externa}
    En este experimento vamos a observar como varia la isoterma cuando modificamos la temperatura de la pared externa del horno.
    Suponemos que cuando mayor sea la temperatura, la isoterma estará mas cerca de la pared esxterna del alto horno. 


    Para ello tomamos dos sistemas con temperaturas externas constantes pero distintos entre si. Graficamos para cada uno de ellos la isoterma 500 y veremos, si la hipotesis es correcta, que en el caso en el que la temperatura externa es mayor, la isoterma esta mas cerca de la pared externa que en el otro caso. 
