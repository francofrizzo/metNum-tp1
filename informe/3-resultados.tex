\section{Resultados}
	
	\subsubsection*{Experimento 4}
	Se observa que cuando hay pocas instancias, buscar la isoterma con el metodo de eliminacion gaussiana o el de factriazacion LU es muy similar. A medida que aumentan las instancias las diferencias entre ambos metodos aumentan, ya que la factorizacion LU se mantiene casi inmutable al aumentar las instancias mientras que la eliminacion gaussiana aumenta notoriamente. Se puede observar que lo que tarda la eliminacion gaussiana cuando tiene 2 instancias es aproximadamentre el doble de lo que tarda cuando es una unica instancia. Cuando son 3 instancias, tarda el triple de lo que tarda cuando es una unica instancia. Entonces saean n instancias entonces lo que tarda la elimimacion gaussiana es lo que tarda cuando es una unica instancia * n. Esto es asi ya que en este metodo numerico, para cada una de las instancias recalcula toda la matriz. En cambio la factorizacion LU, calcula la matriz una unica vez.
  	
	\subsubsection*{Experimento 3}
	Se observa 


  {\color{Gray} Deben incluir los resultados de los experimentos, utilizando el formato más adecuado para su presentación. Deberán especificar claramente a qué experiencia corresponde cada resultado. No se incluirán aquí corridas de máquina.}
