\section{Discusión}


\subsubsection*{Experimento 1: Isoterma según granularidad - seno}
	En este experimento se puede observar que cuando cambiamos los angulos la isoterma se aleja mucho mas de la real que cuando variamos los radios. Esto es asi ya que comparando los errores de dejar de considerar temperaturas y el error de tomar una medida como lienal cuando no lo es, la primera es peor y distorsiona mucho mas la isoterma.
	


\subsubsection*{Experimento 2: Isoterma según granularidad}
	En este caso se observa lo mismo que en el anterior ya que lo unico que estamos variando es que las temperaturas de la pared externa ahora son constantes. Ésto no modifica el error de aproximar la isoterma con menos discretizacion en los angulos o en los radios.



\subsubsection*{Experimento 3: Índice de peligrosidad según granularidad}
  	En el gráfico presentado, se puede notar que no hay diferencia entre los índices de las instancias con 5 y 10 particiones sobre ángulos, ya que al calcularlos utilizando el método explicado anteriormente, el punto más cercano a la pared en la primera, fue considerado en la segunda. 

  	En cambio, en el resto de las instancias, el índice de peligrosidad aumenta, ya que al tomar más particiones, fueron tomados en cuenta nuevos puntos, afectando la isoterma, y por lo tanto, el índice. 

  	De esta forma, demostramos experimentalmente nuestra hipótesis. 


\subsubsection*{Experimento 4}
	En este experimento se puede ver que a medida que disminuimos el grosor de la pared del alto horno la isoterma se encuentra cada vez mas cerca de la pared externa. El indice de peligrosidad mide la relacion entre la temperatura mas alta y la proximidad de la misma a la pared externa. Cuando disminuimos el grosor de la pared, lo que hacemos es acercar la pared interna(donde se encuentra la temperatura mas alta del sistema) a la pared externa. De esta forma la isoterma va a estar mas proxima a la pared externa que cuando el grosor es mayor. Entonces el indice de peligrosidad aumentará, demostrando así lo que se planteó en la hipotesis.
	

\subsubsection*{Experimento 5: Tiempo según número de instancias}
	Se observa que cuando hay pocas instancias, buscar la isoterma con el metodo de eliminacion gaussiana o el de factriazacion LU es muy similar. A medida que aumentan las instancias las diferencias entre ambos metodos aumentan, ya que la factorizacion LU se mantiene casi inmutable al aumentar las instancias mientras que la eliminacion gaussiana aumenta notoriamente. Se puede observar que lo que tarda la eliminacion gaussiana cuando tiene 2 instancias es aproximadamentre el doble de lo que tarda cuando es una unica instancia. Cuando son 3 instancias, tarda el triple de lo que tarda cuando es una unica instancia. Entonces saean n instancias entonces lo que tarda la elimimacion gaussiana es lo que tarda cuando es una unica instancia * n. Esto es asi ya que en este metodo numerico, para cada una de las instancias recalcula toda la matriz. En cambio la factorizacion LU, calcula la matriz una unica vez.


\subsubsection*{Experimento 6: Tiempo según granularidad}
  	En los dos gráficos correspondientes a este experimento, es claro el crecimiento del tiempo de ejecución, a medida que la granularidad aumenta, tanto en los ángulos como en los radios. Esto se debe a la complejidad de los algoritmos utilizados.

  	También, es importante notar como el tiempo de ejecución sobre una sola instancia no cambia demasiado al modificar el método de resolución. Sabemos que, cuando se trata de un solo caso, factorización LU tiene complejidad mayor por constantes que eliminación gaussiana, lo que se puede visualizar en los casos con mayor granularidad de la discretización.


\subsubsection*{Experimento 7: relacion de la isoterma con los bordes}
 	En este experimento se puede ver que la isoterma roja se encuentra mas próxima a la pared interna del horno ya que corresponde a la intancia que tiene menor temperatura en la pared externa (50 grados). Lo mismo ocurre con la isoterma azúl que se encuentra mas cerca de la pared externa debido a que las temperaturas son mas altas (200 grados).

 	Ademas, en uno de los casos la tempreartura externa es la máxima posible y en el otro la mínima. Podemos observar que nunca, con esta discretización, para ningún valor de temperaturas externas válidas, la isoterma puede pasar mas cerca de la pared externa del alto horno que lo que pasa la isoterma marcada en el grafico con azul. Tampoco tampoco es posible que la isoterma pase mas cerca de la pared interna del horno de lo que pasa la isoterma marcada en el grafico con color rojo.


  {\color{Gray} Se incluirá aquí un análisis de los resultados obtenidos en la sección anterior (se analizará su validez, coherencia, etc.). Deben analizarse como mínimo los ítems pedidos en el enunciado. No es aceptable decir que ``los resultados fueron los esperados'', sin hacer clara referencia a la teoría a la cual se ajustan. Además, se deben mencionar los resultados interesantes y los casos ``patológicos'' encontrados.}
