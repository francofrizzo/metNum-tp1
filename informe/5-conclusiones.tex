\section{Conclusiones}

La primer gran conclusión que pudimos sacar fue que el método numérico de factorización LU es mucho más eficiente que el de eliminación gausseana cuando tenemos varias instancias. Para una cantidad chica de instancias son muy similares entre si. Ésto lo pudimos observar mediante los experimentos en los cuales comparabamos los tiempos de ejecución entre ambos métodos para valores de entrada con diferentes cantidades de instancias. 

Otra de las conclusiones fue que cuanto mayor sea la granularidad, la isoterma se aproxima más al valor real. También concluimos que disminuir la cantidad de particiones de ángulos hace que la isoterma se aleje más de la real que disminuir las de los radios. Esto lo vimos tomando como real una isoterma en un sistema muy granularizado, luego disminuimos la granularidad en radios y comparamos entre las isotermas. Después repetimos el experimento pero disminuyendo la cantidad de ángulos.

Finalmente podemos concluir que para lograr una mayor eficiencia en el cálculo de la isoterma debemos utilizar el método de factorizacion LU y utilizar la mayor cantidad posible de particiones.
