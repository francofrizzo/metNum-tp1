\section{Conclusiones}

La primer gran conclusion que pudimos sacar fue que el metodo numerico de factorizacion LU es mucho mas eficiente que el de eliminacion gausseana cuando tenemos varias instancias. Para una cantidad chica de instancias son muy similares entre si. Ésto lo pudimos observar mediante los experimentos en los cuales comparabamos los tiempos de ejecucion entre ambos metodos para valores de entrada con diferentes cantidades de instancias. 
Otra de las conclusiones fue que cuanto mayor sea la granularidad, la isoterma se aproxima mas al valor real. Tambien concluimos que disminuir la cantidad de particiones de angulos hace que la isoterma se aleje mas de la real que disminuir las de los radios. Esto lo vimos tomando como real una isoterma en un sistema muy granularizado, luego disminuimos la granularidad en radios y comparamos entre las isotermas. Despues repetimos el experimento pero disminuyendo la cantidad de angulos.
Finalmente podemos concluir que para lograr una mayor eficiencia en el calculo de la isoterma debemos utilizar el métoso de factorizacion LU y utilizar la mayor cantidad posible de particiones.

  

  {\color{Gray} Esta sección debe contener las conclusiones generales del trabajo. Se deben mencionar las relaciones de la discusión sobre las que se tiene certeza, junto con comentarios y observaciones generales aplicables a todo el proceso. Mencionar también posibles extensiones a los métodos, experimentos que hayan quedado pendientes, etc.}
